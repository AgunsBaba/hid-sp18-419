% status: 0
% chapter: TBD

\title{Google Compute Engine}


\author{Bertolt Sobolik}
\orcid{1234-5678-9012}
\affiliation{%
  \institution{Indiana University}
  \city{Bloomington} 
  \state{Indiana} 
  \postcode{47408}
}
\email{bsobolik@iu.edu}

% The default list of authors is too long for headers}
\renewcommand{\shortauthors}{B. Sobolik}


\begin{abstract}
Google Compute Engine is an Infrastructure as a Service (IaaS)
offering from Google.
\end{abstract}

\keywords{hid-sp18-419, E519, Google, Compute, Engine}


\maketitle

\section{Introduction}

\section{What is IaaS?}

IaaS can be defined as ``a standardized, highly automated offering,
where compute resources, complemented by storage and networking
capabilities are owned and hosted by a service provider and offered to
customers on-demand~\cite{hid-sp18-419-gartneritglossaryiaas}.''
Customers run virtual machines for which they choose the operating
system and storage, but they do not have control over the underlying
hardware or network
infrastructure~\cite{hid-sp18-419-mell2011nist}. IaaS providers
generally differentiate themselves based on pricing, service level
agreements (SLA), storage and networking options, and the types of
database and application development services that they
offer~\cite{hid-sp18-419-perficientchoosingiaas}. Google attempts to
further distinguish their IaaS offering from the offerings of their
competitors with claims of environmental friendliness and greater
flexibility manifest in the availibility of \textit{Preemtive VMs}
which are VMs that automatically terminate after 24 hours which are
offered at a lower cost than other VMs~\cite{hid-sp18-419-gce}.


\section{History}
A preview edition of Google Compute Engine was launched on June 28,
2012~\cite{hid-sp18-419-googleblog20120628} to expand on the
capabilities of Google App Engine which had been around since
2008~\cite{hid-sp18-419-gcp-history-medium}. The preview release
supported Debian and Centos Linux distributions that Google had
customized to some extent. Google Compute Engine was made an official
Google product on December 2, 2013. When it became an official product
support was added for SELinux, CoreOS, SUSE, FreeBSD, and other Linux
distributions. Support was also added for Docker, FOG, xfs, and
aufs. Early customers included Snapchat and
Evite~\cite{hid-sp18-419-googleblog20131202}. The first Google
platform Live event was held in San Francisco on March 25, 2014. At
the second such event on November
4~\cite{hid-sp18-419-googleblog20140812}, 2014 Google announced the
release of Compute Engine Autoscaler which allows developers to
dynamically resize VMs in response to utilization or other triggers,
and the beta release of Local SSD, allowing users to attach local
solid-state drive block storage to VM instances. Some technology
journalists saw this event and its many announcements, including the
launch of Google Container Engine, Google Cloud Interconnect, which
allowed customers to connect directly to Google's data
centers~\cite{hid-sp18-419-gcp-techcrunch-20141104}, and many
enhancements to Google App Engine, as a sign that Google was getting
more serious about competing with Amazon in the cloud computing
market~\cite{hid-sp18-419-forbes-gcp-20141104}.

In January, 2015, Local SSD became generally available and Google
Cloud Monitoring, including support for Compute Engine went into
beta~\cite{hid-sp18-419-googleblog20150113}. This came with the
integration of Stackdriver, which Google acquired in May of
2014~\cite{hid-sp18-419-googleblog20140507}. The beta release of
Preemptive VMs happend in May, 2015 followed by the official release
in September of the same
year~\cite{hid-sp18-419-venturebeat-preemptive-vms}. Citadel and
Descartes Labs were early adopters of the
technology~\cite{hid-sp18-419-googleblog20150518}. In July, 2015
Google added Windows Server to the VMs available on Compute
Engine~\cite{hid-sp18-419-googleblog20150715}, which had been in beta
since December, 2014~\cite{hid-sp18-419-googleblog20141208}. This
enabled them to compete more effectively with Microsoft Azure IaaS,
which at the time was much closer to Amazon EC2 in overall IaaS market
share than discussed
below~\cite{hid-sp18-419-statista-iaas-market2015}.

The beta release of Nvidia Tesla K80 GPUs on Google Compute Engine was
announced in February of 2017, allowing customers to attach up to
eight GPUs to a custom VM. At the same time, Intel Skylake processors
were made available in a number of
regions.~\cite{hid-sp18-419-googleblog20170221}.

In March, 2017 Google enhanced their support of Microsoft SQL Server
on GCP, adding SQL Server Enterprise images to the VMs available along
with a number of other relational database
announcements~\cite{hid-sp18-419-googleblog20170309}.

In May of 2017, Google expanded their machine types available as
Compute Engine VMs to include up to 64 virtual processors and up to
416 GB of memory which are certified for running SAP's in-memory
database HANA. In October, 2017 Google increased their top offering to
96 CPUs and 624 GB of memory, and they are reported to be working on
products that will deliver 4 TB of
memory~\cite{hid-sp18-419-techcrunch-gce-20171005}.

In February, 2018 Google added the ability to create VMs from
templates based on existing VM
instances~\cite{hid-sp18-419-googleblog20180222}.

\section{Locations}
One thing that made it difficult for Google to compete effectively
with other players in the IaaS space was how few regional data centers
they had relative to their competitors. From the beta launch of the
service in 2012 until mid-2014, they only had two locations,
us-central1 in Iowa and europe-west1 in
Belgium~\cite{hid-sp18-419-gcp-history-medium}. Expansion to the
Asia-Pacific region was announced on April 14, 2014 with the openning
of asia-east1 in Taiwan to support customers in the region including
Applibot, an mobile gaming company in Japan, and Tagtoo, an content
tagging startup in Taiwan. Japanese and Traditional Chinese web sites
and developer consoles were launched at the same
time~\cite{hid-sp18-419-googleblog20140414}. They started offering
services to GCP customers from an additional sites, us-east1 in South
Carolina, in October, 2015. Before this the site had only been used
for Google's internal applications. At the time us-east1 was launched
Microsoft had 20 locations hosting Azure
services~\cite{hid-sp18-419-techcrunch-gcp-20151001}. Another location,
us-west1 in Oregon was made available to GCP customers in July, 2016,
and a Tokyo location was announced in March, 2016 and went live in
November 2016s~\cite{hid-sp18-419-googleblog20161108}. The Oregon
location was seen as critical to making their service attractive to
the online gaming community. They anticipated a 30-80\% reduction in
latency for customers located in cities on the west coast of the US
with the launch of us-west1~\cite{hid-sp18-419-googleblog20160720}.

In September, 2016 Google announced eight additional
locations~\cite{hid-sp18-419-gcp-history-medium}:
\begin{itemize}
  \item Mumbai
  \item Singapore
  \item Sydney
  \item Northern Virginia (launched May 2017)
  \item São Paulo
  \item London (launched July 2017)
  \item Finland
  \item Frankfurt
\end{itemize}

Currently, Google has 15 locations on five contenents: five in North
America, four in Europe, four in Asia, and one each in South America
and Australia. Each region is broken out into between two and four
zones. Certain resources are only available in certain zones within a
location, and some resources must be in the same zone. For example, a
disk must be in the same zone as the instance to which it is attached,
and GPUs are only available in certain zones within certain
regions~\cite{hid-sp18-419-gce-regions-zones}.


\section{Virtual Machine Types}

Google offers 25 different standard machine types. Machine types are
defined by the number of virtual CPUs, gigabytes of memory, maximum
number of persistent disks, and maximum total size of persistent disk
storage. The two smallest instances, f1-micro and g1-small, run on .2
or .5 fractions of a shared physical core, have .6 or 1.7 GB of memory
and are limited to 4 persistent disks totaling 3 terrabytes. The other
23 machine types are named with the prefix n1. Customers can choose
from five different Intel Xeon E5 processors for these instances
ranging from the the older 2.6 GHz Sandy Bridge to the
fifth-generation 2.0 GHz Skylake. Depending on the region and zone of
the VM certain processors and configurations are not available. The n1
prefix VMs are organized into three groups: standard, high-memory, and
high-cpu. There is a single CPU option for the standard configuration
which has 3.75 GB of memory. The other 22 options are configured with
2-96 CPUs and differ from each other by the amount of memory ranging
from 1.8GB for the n1-highcpu-2 configuration to 1.4TB for the
n1-megamem-96. Up to 16 persistant disks adding up to a total of 16 TB
of persistent storage can be attached to any of the n1 prefixed VMs,
which can be augmented with additional storage from Google Cloud
Platform's variety of database and blob storage options. Custom
machine types are also available, and up to 128 persistent disks can
be attached in beta VM configurations
~\cite{hid-sp18-419-gce-machine-types}.

\section{Competitors and Market Share}

Google Compute Engine's main competitors are Amazon Elastic Cloud
Compute (EC2), Microsoft Azure IaaS, AlibabaCloud Elastic Compute
Service, and Rackspace Public Cloud. Amazon is by far the leader in
this space with 44.2\% of total revenue in 2016 (almost \$10
billion). In terms of revenue, Google Compute Engine was the fourth
largest IaaS public cloud service, behind Amazon, Microsoft, and
Alibaba but ahead of Rackspace due to year over year growth from
2015. In 2016, GCP's revenue doubled from \$250 million in 2015
to \$500 million. GCP had a 2.3\% share of the IaaS market in 2016 and
is growing faster than all of the above named competitors except for
Alibaba, which grew 126.5\% in
2016~\cite{hid-sp18-419-gartnerpr2017}.


\section{Conclusion}

Put here an conclusion. Conclusion and abstracts must not have any
citations in the section.


\begin{acks}

  The authors would like to thank Dr.~Gregor~von~Laszewski for his
  support and suggestions to write this paper.

\end{acks}

\bibliographystyle{ACM-Reference-Format}
\bibliography{report} 

