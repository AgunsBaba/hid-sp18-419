\section{Neptune}\index{Neptune}\index{AWS}\index{Graph database}\index{NoSQL}\index{Amazon}\index{SPARQL}\index{Apache!TinkerPop Gremlin}\index{WC3}\index{Resource Description Framework}

{\bf the extensive number of backslash index refernces is confusing as the main
  text is not about microsoft for example. Improve in all you rentries
and select the once you realy need to index. just put the index bellow
the section so the text stays readable in the source.}

Neptune is a graph database\index{Graph database} service that was
announced at the AWS\index{AWS}Re:INVENT conference in November of
2017\cite{hid-sp18-419-www-tc_neptune}. Graph databases are
NoSQL\index{NoSQL}databases that used graph structures to organize
data\cite{hid-sp18-419-www-tp-graph-db}. They are commonly used for
social networking applications, but can be used for recommendation
engines, logistics, and other applications. Amazon\index{Amazon}
offers Neptune as a fully managed product. It supports Apache
TinkerPop Gremlin\index{Apache!TinkerPop Gremlin} and
SPARQL\index{SPARQL} open source graph APIs. One can choose Gremlin or
the W3C\index{WC3} standard Resource Description
Framework\index{Resource Description Framework}
model\cite{hid-sp18-419-www-aws-neptune}.
